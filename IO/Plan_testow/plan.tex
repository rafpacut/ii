\documentclass[11pt,a4paper]{article}
% Interlinia
\linespread{1.15}
% Polskie różności
\usepackage[utf8]{inputenc}
\usepackage{polski}
%\usepackage[polish]{babel}
% Kropki po numerach sekcji w tekście...
\usepackage{secdot}
\sectiondot{subsection}
\sectiondot{subsubsection}
% ... i spisie treści
\usepackage{tocloft}
\renewcommand{\cftsecaftersnum}{.}
\renewcommand{\cftsubsecaftersnum}{.}
\renewcommand{\cftsubsubsecaftersnum}{.}
% Przeklęte tabelki
\usepackage[labelfont=it]{caption}
\setcounter{table}{-1}
% do rysowania wykresu gantta
\usepackage{pgfgantt}
% równanie marginesów przy harmonogramie
\usepackage[a4paper]{geometry}

\begin{document}

%
% Strona tytułowa
%
\pagenumbering{gobble}
\begin{center}
	\Large
	Studencka Pracownia Inżynierii Oprogramowania \\[0.5cm]
	Instytut Informatyki Uniwersytetu Wrocławskiego 

	\vspace*{\fill}
	Rafał Pacut, Gabriel Sztorc \\[1cm]
	{\Huge System obsługi bibliotek publicznych} \\[1cm]
	Plan Testów
	\vspace*{\fill}

	Wrocław, 4 listopada 2014 \\[0.5cm]
	Wersja 0.3
\end{center}
\newpage
\pagenumbering{arabic}
\setcounter{page}{2}
%
% Koniec strony tytułowej
%

%
% Tabela zmian
%
\begin{table}
\caption{Historia zmian dokonanych w dokumencie}
\begin{tabular}{|l|l|l|l|}
    \hline
    \multicolumn{1}{|c|}{Data} & \multicolumn{1}{c|}{Numer wersji} & 
        \multicolumn{1}{c|}{Opis} & \multicolumn{1}{c|}{Autor} \\
    \hline \hline
    2015-01-15 & 0.1 & Utworzenie dokumentu & Gabriel Sztorc \\
    \hline
    2014-01-17 & 0.2 & Rozwinięcie dokumentu & Rafał Pacut \\
    \hline
    2014-01-17 & 0.3 & Uzupełnienie dokumentu & Gabriel Sztorc \\
    \hline
    2014-01-19 & 0.4 & Redakcja i korekta & Gabriel Sztorc \\
    \hline
\end{tabular}
\end{table}
%
% Koniec tabeli zmian
%

% \section*{Spis treści}
\tableofcontents

\newpage

\section{Wstęp}
Przedmiotem tego dokumentu jest szczegółowy opis procesu testowania systemu obsługi bibliotek publicznych.
Opisuje on, które funkcje będą podlegały testowaniu i kryteria oceny wyników testów.
Oprócz tego zawiera on listę zasobów niezbędnych do przeprowadzenia testów i harmonogram
zgodnie z którym będą one przeprowadzone.

\subsection{Definicje}

	\begin{enumerate}
		\item Użytkownik - klient biblioteki. 
		\item Pracownik - pracownik biblioteki. 
		\item Wada - każda nieprawidłowość wykonania funkcji, w wyniku której może dojść do awarii.
		\item Awaria - każda wada powodująca nieprawidłowe działanie systemu. 
	\end{enumerate}

\section{Elementy testowane i zakres testów}
Testom zostaną poddane następujące moduły systemu:
	\begin{itemize}
		\item Baza danych,
		\item Funkcje użytkownika,  %TODO: kto to jest użytkownik
		\item Funkcje pracownika,   % 	i kto to jest pracownik.
		\item Interfejs graficzny;
	\end{itemize}

	\subsection{Testowanie obciążeniowe i wydajnościowe}

	W ramach testu wygenerowana zostanie bardzo duża ilość zapytań do bazy danych, \\
	Zasymulowane zostanie również logowanie i wchodzenie na stronę główną bardzo dużej ilości użytkowników. \\ 
	\\
	Przy każdym z powyższych testów mierzony będzie czas wykonania 
	tj. czas między zapytaniem wysłanym przez skrypt do otrzymania odpowiedzi. \\


	\subsection{Testy spójności}

	Przeprowadzone zostaną testy na spójność komponentów  \\
	i wykorzystanie wszystkich składowych systemu jednocześnie.\\

	\subsection{Spełnienie założeń}

	Skonstruowany zostanie szkic wykonania funkcji systemu zgodnie 
	z dokumentacją, w oparciu o przypadki użycia.\\
	Następnie funkcje będą testowane zgodnie ze szkicem.

	\subsection{Testowanie zgodności}

	Testowanie prawidłowego działania strony WWW na różnych przeglądarkach i platformach.\\
	 W szczególności sprawdzane będzie prawidłowe zachowanie
	interfejsu użytkownika i zgodność wszystkich podstron ze standardem W3.

	\subsection{Testowanie bezpieczeństwa}

	Przeprowadzone zostaną testy dotyczące nieprawidłowego użytkowania strony WWW przez użytkownika, takie jak:
		\begin{itemize}

			\item	 zmiana adresu url, 
			\item używanie znaków specjalnych przy zapytaniach
				do bazy danych np. \%s, 
			\item  próby SQL injection; 
		\end{itemize}
	W ramach tych testów sprawdzony zostanie system automatycznego tworzenia kopii zapasowych i modułu odpowiedzialnego za odzyskiwanie stanu systemu z kopii zapasowej.\\

\section{Kryteria przejścia testu}
	Za moduł, który przeszedł testy uznaje się moduł, na którym:
		\begin{itemize}
			\item przeprowadzone zostały wszystkie przypadki testowe,
			\item conajwyżej 10\% testów wykazało pomniejsze wady, %TODO: co to jest pomniejsza wada
			\item nie wystąpiła żadna awaria;				% co to jest awaria.
		\end{itemize}

\section{Kryteria zawieszenia i wznowienia}
Proces testowania powinien zostać zawieszony, gdy wystąpi którakolwiek z następujących sytuacji:

	\begin{itemize}
		\item  istnieje nieprzetestowana funkcja zależna od funkcji, której conajmiej jeden test wykazał awarię,
		\item  istnieje nieprzetestowana funkcja zależna od funkcji, której liczba testów wykazujących wadę przekroczyła 30\%;
	\end{itemize}
Wznowienie procesu testowania może rozpocząć się jedynie gdy poprawiona wersja modułu przejdzie wszystkie testy wykonane do momentu zawieszenia.

\section{Produkty testowania}
W skład produktów testowania wchodzą:
	\begin{itemize}
		\item dokumentacja,
		\item przypadki testowe,
		\item narzędzia używane do testowania,
		\item skrypty symulujące,
		\item dzienniki błedów,
	\end{itemize}

\section{Środowiska testowania}
Do testowania potrzebne będą conajmniej dwa komuputery z systemami operacyjnymi Windows XP, Windows 7, Windows 8.1, Linux i Makintosh.
Jeden ze stosunkowo starym sprzętem i jeden ze sprzętem klasy przeciętnej.
Na komputerach przetestowane zostaną przeglądarki: Safari, Chrome, Firefox i IE.

\section{Potrzeby szkoleniowe}
Do wykonania testów potrzebny będzie wykwalifikowany zespół testerów. Ich zadaniem będzie obsługa programów jak Webserver Stress Tool, 
napisanie i obsługa skryptów emulujących wielu użytkowników itp.

\section{Harmonogram}
Testom zostanie poddany każdy moduł po zakończeniu pojedyńczego cyklu wytwarzania oprogramowania.
Pozytywny wynik testów jest wymagany do rozpoczęcia kolejnego cyklu.
Po zakończeniu wszystkich faz wytwarzania oprogramowania testom poddana zostanie całość systemu.

























\end{document}
