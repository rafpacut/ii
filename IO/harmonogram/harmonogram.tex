\documentclass[11pt,a4paper]{article}
% Interlinia
\linespread{1.15}
% Polskie różności
\usepackage[utf8]{inputenc}
\usepackage{polski}
%\usepackage[polish]{babel}
% Kropki po numerach sekcji w tekście...
\usepackage{secdot}
\sectiondot{subsection}
\sectiondot{subsubsection}
% ... i spisie treści
\usepackage{tocloft}
\renewcommand{\cftsecaftersnum}{.}
\renewcommand{\cftsubsecaftersnum}{.}
\renewcommand{\cftsubsubsecaftersnum}{.}
% Przeklęte tabelki
\usepackage[labelfont=it]{caption}
\setcounter{table}{-1}
% do rysowania wykresu gantta
\usepackage{pgfgantt}
% równanie marginesów przy harmonogramie
\usepackage[a4paper]{geometry}
%tikz
\usepackage{tikz}
\usepackage{tikz-uml}


\begin{document}

%
% Strona tytułowa
%
\pagenumbering{gobble}
\begin{center}
	\Large
	Studencka Pracownia Inżynierii Oprogramowania \\[0.5cm]
	Instytut Informatyki Uniwersytetu Wrocławskiego 

	\vspace*{\fill}
	Rafał Pacut, Gabriel Sztorc \\[1cm]
	{\Huge System obsługi bibliotek publicznych} \\[1cm]
	Harmonogram i Przypadki Użycia
	\vspace*{\fill}

	Wrocław, 29 listopada 2014 \\[0.5cm]
	Wersja 0.3
\end{center}
\newpage
\pagenumbering{arabic}
\setcounter{page}{2}
%
% Koniec strony tytułowej
%

%
% Tabela zmian
%
\begin{table}
\caption{Historia zmian dokonanych w dokumencie}
\begin{tabular}{|l|l|l|l|}
    \hline
    \multicolumn{1}{|c|}{Data} & \multicolumn{1}{c|}{Numer wersji} & 
        \multicolumn{1}{c|}{Opis} & \multicolumn{1}{c|}{Autor} \\
    \hline \hline
    2014-10-20 & 0.1 & Utworzenie dokumentu & Rafał Pacut, Gabriel Sztorc \\
    \hline
\end{tabular}
\end{table}
%
% Koniec tabeli zmian
%

% \section*{Spis treści}
\tableofcontents

\newpage


{\newgeometry{left=1cm, right=1.5cm}
	\section{Harmonogram Prac}
\ganttset{%
	calendar week text=
	{
		\currentweek
	}
	}%
	\begin{ganttchart}
		[
			vgrid={draw=none, draw=none, draw=none,  dotted},
			hgrid,
			x unit=0.8mm,
			y unit chart=0.87cm,
			time slot format=isodate
		]{2014-11-9}{2015-03-20}
	\gantttitlecalendar{year, month, week} \\

	\ganttgroup{Group 1}{2014-11-09}{2015-03-12} \\ 
	\ganttbar{Impl. bazy danych}{2014-11-09}{2015-01-20}  \\
	\ganttbar{Poprawa bazy danych}{2015-01-21}{2015-02-01} \\
	\ganttbar{Impl. systemu dla pracowników}{2015-02-02}{2015-02-20}  \\
	\ganttbar{Poprawa systemu}{2015-02-21}{2015-03-12} \\

	\ganttgroup{Group 2}{2014-11-9}{2015-02-28} \\
	\ganttbar{Impl. platformy webowej}{2014-11-09}{2014-12-28} \\
	\ganttbar{Poprawa platformy}{2014-12-29}{2015-01-14} \\
	\ganttbar{Scalanie platformy i systemu z bazą danych}{2015-02-15}{2015-02-28} \\

	\ganttgroup{Group 3}{2014-12-28}{2015-03-12} \\
	\ganttbar{Testy platformy webowej}{2014-12-28}{2015-01-14} \\
	\ganttbar{Testy bazy danych}{2015-01-20}{2015-02-01} \\
	\ganttbar{Testy systemu dla pracowników}{2015-02-20}{2015-03-12} \\


	\ganttlink{elem1}{elem11}
	\ganttlink{elem1}{elem2}

	\ganttlink{elem2}{elem3}
	\ganttlink{elem3}{elem4}
	\ganttlink{elem3}{elem12}


	\ganttlink{elem6}{elem10}
	\ganttlink{elem6}{elem7}
	\ganttlink{elem7}{elem8}



\end{ganttchart}

\section{Przypadki Użycia}

\begin{tikzpicture}


		\begin{umlsystem}[x=9, fill=green!10]{System}
			\umlusecase[x=-1]{przeglądnięcie repertuaru}
			\umlusecase[x=-3, y=-1, width=8ex]{rejestracja}
			\umlusecase[x=-1, y=-2]{wypożyczenie książki}
			\umlusecase[x=-1, y=-3.25]{Sprawdzenie dostępności pozycji}
			\umlusecase[x=-1, y=-5.75]{wyświetlenie listy zalegających}

			\umlusecase[x=3, y=-1, fill=red!10]{Zarządzanie kontami}
			\umlusecase[x=3, y=-4.25, fill=red!10]{Zarządzanie bazą danych}

		\end{umlsystem}

		\umlactor[y=-0.5]{użytkownik}
		\umlactor[x=3, y=-2, scale=0.7]{ogólny}
		\umlactor[y=-5]{pracownik}
		\umlactor[x=17, y=-2]{administrator}

		%relations:
		\umlassoc{użytkownik}{usecase-1}
		\umlVHextend[pos stereo=1.5]{użytkownik}{ogólny}

		\umlassoc{ogólny}{usecase-2}
		\umlassoc{ogólny}{usecase-3}
		\umlassoc{ogólny}{usecase-4}

		\umlVHextend[pos stereo=1.5]{pracownik}{ogólny}
		\umlCNassoc{pracownik}{2,-5}{usecase-5}

		\umlassoc{administrator}{usecase-6}
		\umlassoc{administrator}{usecase-7}

		


\end{tikzpicture}

\restoregeometry






















\end{document}
