\documentclass[11pt,a4paper]{article}
% Interlinia
\linespread{1.15}
% Polskie różności
\usepackage[utf8]{inputenc}
\usepackage{polski}
%\usepackage[polish]{babel}
% Kropki po numerach sekcji w tekście...
\usepackage{secdot}
\sectiondot{subsection}
\sectiondot{subsubsection}
% ... i spisie treści
\usepackage{tocloft}
\renewcommand{\cftsecaftersnum}{.}
\renewcommand{\cftsubsecaftersnum}{.}
\renewcommand{\cftsubsubsecaftersnum}{.}
% Przeklęte tabelki
\usepackage[labelfont=it]{caption}
\setcounter{table}{-1}
% do rysowania wykresu gantta
\usepackage{pgfgantt}
% równanie marginesów przy harmonogramie
\usepackage[a4paper]{geometry}

\begin{document}

%
% Strona tytułowa
%
\pagenumbering{gobble}
\begin{center}
	\Large
	Studencka Pracownia Inżynierii Oprogramowania \\[0.5cm]
	Instytut Informatyki Uniwersytetu Wrocławskiego 

	\vspace*{\fill}
	Rafał Pacut, Gabriel Sztorc \\[1cm]
	{\Huge System obsługi bibliotek publicznych} \\[1cm]
	Analiza wstępna
	\vspace*{\fill}

	Wrocław, 4 listopada 2014 \\[0.5cm]
	Wersja 0.3
\end{center}
\newpage
\pagenumbering{arabic}
\setcounter{page}{2}
%
% Koniec strony tytułowej
%

%
% Tabela zmian
%
\begin{table}
\caption{Historia zmian dokonanych w dokumencie}
\begin{tabular}{|l|l|l|l|}
    \hline
    \multicolumn{1}{|c|}{Data} & \multicolumn{1}{c|}{Numer wersji} & 
        \multicolumn{1}{c|}{Opis} & \multicolumn{1}{c|}{Autor} \\
    \hline \hline
    2014-10-20 & 0.1 & Utworzenie dokumentu & Rafał Pacut, Gabriel Sztorc \\
    \hline
    2014-10-28 & 0.2 & Rozwinięcie dokumentu & Gabriel Sztorc \\
    \hline
    2014-11-04 & 0.3 & Uzupełnienie dokumentu & Gabriel Sztorc \\
    \hline
    2014-11-13 & 0.4 & Harmonogram & Rafał Pacut \\
    \hline
\end{tabular}
\end{table}
%
% Koniec tabeli zmian
%

% \section*{Spis treści}
\tableofcontents

\newpage

\setcounter{section}{-1}
\section{Streszczenie dokumentu}
Przedmiotem niniejszego dokumentu jest analiza wstępna projektu oprogramowania
wspomagającego pracę bibliotek publicznych. Przedstawione zostają: cele i
zakres projektu, wstępny opis i analiza proponowanego rozwiązania oraz zarys
wymagań.

\section{Wstęp}
\subsection{Cele przedsięwzięcia}
Celem przedsięwzięcia jest wytworzenie oprogramowania które usprawni
wykonywanie codziennych czynności związanych z pracą bibliotek publicznych
Oprogramowanie powino ułatwić czytelnikom uzyskiwanie informacji o zbiorach
biblioteki i ich dostępności, przyśpieszyć obsługę czytelników przez personel
biblioteki, ułatwić wykonywanie prac ewidencyjnych oraz usprawnić egzekwowanie
przepisów biblioteki dotyczących terminów zwrotów i limitów wypożyczeń. 

\subsection{Zakres przedsięwzięcia}
W ramach projektu wykonane zostaną następujące prace:
\begin{itemize}
  \item Wytworzenia oprogramowania do obsługi wypożyczeń, zwrotów i innych
    czynności wykonywanych przez pracowników biblioteki, mogącego
    pracować na wielu stanowiskach jednocześnie. 
  \item Wytworzenie oprogramowanie do przeglądania i przeszukiwania katalogu
    zbiorów bibliotecznych przeznaczone dla czytelników, także mogącego
    pracować na wielu stanowiskach równocześnie oraz oferującego dostęp przez
    Internet.
  \item Wytworzenie systemu do przechowywania informacji o zbiorach
    bibliotecznych, czytelnikach, wypożeczeniach oraz udostępniania tych
    informacji za pośrednictwem sieci komputerowej.
  \item Przygotowanie procesu instalowania i konfigurowania systemu oraz
    przeszkolenia użytkowników, który umożliwi wdrożenie oprogramowania w wielu
    bibliotekach publicznych.
\end{itemize}

\subsection{Definicje}
Poniżej przedstawione zostają definicje niektórych pojęć używanych w
dokumencie:
\begin{description}
  \item[czytelnik] Użytkownik systemu będący zarejestrowanym czytelnikiem
    biblioteki.
  \item[klient] Program, z którego korzystają użytkownicy końcowi systemu,
    komunikujący się z serwerem.
  \item[pracownik] Użytkownik systemu należący do personelu biblioteki.
  \item[serwer] Komputer i uruchomione na nim oprogramowanie przechowujące w
    scentralizowany sposób dane systemu i udostępniające je klientom za
    pośrednictwem sieci komputerowej.
  \item[zbiory] Dzieła zgromadzone w bibliotece, udostępniane czytelnikom.
\end{description}

\section{Opis proponowanego rozwiązania}
System wykonany zostanie w architekturze klient-serwer. Część serwerowa będzie
realizowała funkcje przechowywania i dystrybucji danych. Użytkownicy będą
komunikować się z systemem za pomocą oprogramowania klienta.

\subsection{Elementy systemu}
\begin{itemize}
  \item Centralnym elementem systemu będzie baza danych przechowująca
    informacje o: zbiorach biblioteki, zarejestrowanych czytelnikach,
    użytkownikach systemu i ich uprawnieniach, historii wypożyczeń i zwrotów.
  \item Funkcje systemu przewidziane dla czytelników będą udostępniane za
    pośrednictwem strony WWW. Część serwerowa serwisu WWW będzie
    komunikować się z bazą danych i generować dynamicznie kod strony w
    odpowiedzi na akcje użytkownika.
  \item Na część kliencką serwisu WWW składać się będą elementy graficzne
    strony, szablony kodu HTML i CSS, do których wstawione mają być dane
    uzyskane z bazy oraz kod w języku JavaScript.
  \item Pracownicy biblioteki korzystać będą z oddzielnego programu klienta,
    komunikującego się bezpośrednio z bazą danych.
\end{itemize}

\subsection{Środowisko systemu}
Projekt będzie wykorzystywał system zarządzania bazą danych {\sc PostgreSQL}.
Wszystkie jego elementy wykonujące się na sprzęcie biblioteki będą działać pod
kontrolą systemu operacyjnego \mbox{\sc Linux}. Strona WWW powinna działać w
najnowszych wersjach popularnych przeglądarek \mbox{\sc Chrome}, \mbox{\sc
Firefox} i {\sc \mbox{Internet} \mbox{Explorer}}.

\section{Ogólny opis wymagań}
W tym rozdziale przedstawiony zostaje zarys wymagań, które powinien spełniać
system. Wymienione zostają rodzaje danych jakie system powinien przechowywać, a
następnie czynności jakie powinni móc wykonać użytkownicy systemu.

\subsection{Przechowywane dane}
W podrozdiale tym wymienione są dane jakie system powinien przechowywać. Lista
ta nie wyklucza przechowywania dodatkowych informacji, na przykład możliwości
dodania uwag i komentarzy do niektórych wpisów. Poniżej wymieniony zostaje
minimalny zestaw danych, które muszą być przechowywane by system mógł
spełniać swoje zadania:

\begin{itemize}
  \item Lista czytelników zarejestrowanych w bibliotece. Dla każdego czytelnika
    powinny być przechowywane informacje identyfikacyjne oraz dane kontaktowe.
  \item Wykaz zbiorów biblioteki. Dla każdego dzieła winny być przechowywany
    kompletny zestaw danych pozwalających je zidentyfikować, numer ewidencyjny
    używany przez system oraz rodzaj oraz gatunek dzieła. Powinna istnieć
    możliwość wydzielenia materiałów wypożyczanych od dostępnych wyłącznie w
    ramach czytelni. Powinna także istnieć możliwość odnotowywania informacji o
    zmianach w zbiorach wynikających z zagubień lub zniszczeń. 
  \item Lista pracowników korzystających z systemu. Dla każdego powinna
    być przechowywana unikalna nazwa użytkownika identyfikująca pracownika w
    systemie.
  \item Informacje o wypożyczeniach bieżących i minionych: ile razy dana
    pozycja była wypożyczana, przez kogo, daty wypożyczeń i zwrotów.
\end{itemize}

\subsection{Funkcje dostępne dla czytelników}
Czytelnik powinien mieć dostęp do elektronicznego katalogu zbiorów
bibliotecznych. Powinna istnieć możliwość przeszukiwania katalogu według
złożonych kryteriów. Czytelnik powinien być poinformowany o dostępności
wskazanych dzieł. Powinien istnieć mechanizm rezerwacji, który mógłby być
włączony lub wyłączony w konfiguracji systemu, w zależności od organizacji
pracy biblioteki.

\subsection{Funkcje dostępne dla pracowników}
Oprogramowanie używane przez pracowników powinno umożliwiać wykonanie
następujących czynności: 
\begin{itemize}
  \item Zarejestrowanie nowego czytelnika w systemie.
  \item Wprowadzenie informacji identyfikujących pozycję ze zbiorów i
    oznaczenie jej jako wypożyczonej przez określonego czytelnika.
  \item Wprowadzenie informacji o oddaniu wypożyczonego dzieła.
  \item Dodanie nowej pozycji do wykazu zbiorów biblioteki.
  \item Uzyskanie informacji o czytelnikach preztrzymującyh wypożyczone dzieła
    powyżej regulaminowego limitu.
  \item Przeprowadzenie inwentaryzacji zasobów bibliotecznych.
\end{itemize}

\subsection{Dodatkowe wymagania}
System powinien wykorzystywać fizyczne znaczniki takie jak kody kreskowe,
mogące być umieszczone zarówno na przechowywanych w bibliotece materiałach, jak
i na kartach bibliotecznych czytelników i pozwalające usprawnić proces
wprowadzania danych identyfikacyjnych za pomocą automatycznych czytników. 

W zakresie obsługiwanych czynności ewidencyjnych system powinien
umożliwić użytkownikom działanie zgodne z Rozporządzeniem Ministra Kultury i
Dziedzictwa Narodowego z dnia 29 października 2008 r. w sprawie sposobu
ewidencji materiałów bibliotecznych~\cite{rozpo}.

\newpage
%\section{Harmonogram}
%Nie wiem dlaczego '\section{Harmonogram}' w tym pliku powodowało 
%pustą linię. Przeniosłem go do gantt.tex
\include{gantt}

\begin{thebibliography}{99}
  \bibitem{rozpo} \textit{Rozporządzenie Ministra Kultury i Dziedzictwa
    Narodowego z dnia 29 października w sprawie sposobu ewidencji materiałów
    bibliotecznych.} Dziennik Ustaw 2008 nr 205 poz. 1283.  

\end{thebibliography}

\end{document}
